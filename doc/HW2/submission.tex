\documentclass[11pt]{article}
\usepackage{subfigure,wrapfig,graphicx,booktabs,fancyhdr,amsmath,amsfonts}
\usepackage{bm,amssymb,amsmath,amsthm,wasysym,color,fullpage,setspace,multirow}
\usepackage{listings, xcolor}
\usepackage{pdfpages}
\newcommand{\vb}{\boldsymbol}
\newcommand{\vbh}[1]{\hat{\boldsymbol{#1}}}
\newcommand{\vbb}[1]{\bar{\boldsymbol{#1}}}
\newcommand{\vbt}[1]{\tilde{\boldsymbol{#1}}}
\newcommand{\vbs}[1]{{\boldsymbol{#1}}^*}
\newcommand{\vbd}[1]{\dot{{\boldsymbol{#1}}}}
\newcommand{\vbdd}[1]{\ddot{{\boldsymbol{#1}}}}
\newcommand{\by}{\times}
\newcommand{\tr}{{\rm tr}}
\newcommand{\cpe}[1]{\left[{#1} \times \right]}
\newcommand{\sfrac}[2]{\textstyle \frac{#1}{#2}}
\newcommand{\ba}{\begin{array}}
\newcommand{\ea}{\end{array}}
\renewcommand{\earth}{\oplus}
\newcommand{\sinc}{{\rm \hspace{0.5mm} sinc}}
\newcommand{\tf}{\tilde{f}}
\newcommand{\tbox}[1]{\noindent \fbox{\parbox{\textwidth}{#1}}}
\DeclareMathAlphabet{\mathpzc}{OT1}{pzc}{m}{it}
\definecolor{mylilas}{RGB}{170,55,241}
\definecolor{mygreen}{RGB}{0,168,45}

\title{ASE 389P.4 Methods of Orbit Determination \\ Homework 2}
\author{Corey L Marcus} \date{Thursday, February 18\textsuperscript{th}}

%command to write C++ nicely
\def\CC{{C\nolinebreak[4]\hspace{-.05em}\raisebox{.4ex}{\tiny\bf ++}}}

%commands to include C++ code in appendix
\lstset { %
	language=C++,
	backgroundcolor=\color{black!5}, % set backgroundcolor
	basicstyle=\tiny,% basic font setting
}

\begin{document}
\onehalfspace
\maketitle

\abstract{abstract}

\section{Introduction}

intro

\section{Problem One}

\subsection{a}

Derivation of J2...

\subsection{b}

Plot the orbital elements

\begin{figure}[h!]
	\centering
	\includegraphics[clip,trim=3.25cm 8.5cm 4.25cm 8.5cm, width=\linewidth]{figs/OE.pdf}
	\caption{caption}
	\label{fig:OE}
\end{figure}

\begin{figure}[h!]
	\centering
	\includegraphics[clip,trim=3.25cm 8.5cm 4.25cm 8.5cm, width=.5\linewidth]{figs/OrbitPeriod.pdf}
	\caption{caption}
	\label{fig:OrbitPeriod}
\end{figure}

\subsection{c}

Energy

\begin{figure}[h!]
	\centering
	\includegraphics[clip,trim=3.25cm 8.5cm 4.25cm 8.5cm, width=.5\linewidth]{figs/energy.pdf}
	\caption{caption}
	\label{fig:energy}
\end{figure}

\subsection{d}

Angular momentum

\section{Problem 2}

\subsection{a}

Energy

\begin{figure}[h!]
	\centering
	\includegraphics[clip,trim=3.25cm 8.5cm 4.25cm 8.5cm, width=.5\linewidth]{figs/energydrag.pdf}
	\caption{caption}
	\label{fig:energydrag}
\end{figure}

\subsection{b}

Change in orbital elements

\begin{figure}[h!]
	\centering
	\includegraphics[clip,trim=3.25cm 8.5cm 4.25cm 8.5cm, width=\linewidth]{figs/dOE.pdf}
	\caption{caption}
	\label{fig:dOE}
\end{figure}

\newpage
\appendix
\section{Code}

\subsection{\texttt{main.cc}}
\lstinputlisting{../../src/HW2/main.cc}

\subsection{\texttt{VehicleState.h}}
\lstinputlisting{../../src/HW2/VehicleState.h}

\subsection{\texttt{Util.h}}
\lstinputlisting{../../src/HW2/Util.h}

\subsection{\texttt{VehicleState.cc}}
\lstinputlisting{../../src/HW2/VehicleState.cc}

\subsection{\texttt{Util.cc}}
\lstinputlisting{../../src/HW2/Util.cc}

%plot matlab files now
\lstset { %
	language=matlab,
	backgroundcolor=\color{black!5}, % set backgroundcolor
	basicstyle=\tiny,% basic font setting
}

\subsection{\texttt{SymbolicPartials.m}}
\lstinputlisting{../../src/HW2/SymbolicPartials.m}

\subsection{\texttt{plotting.m}}
\lstinputlisting{../../src/HW2/plotting.m}


\end{document}
