\documentclass[11pt]{article}
\usepackage{subfigure,wrapfig,graphicx,booktabs,fancyhdr,amsmath,amsfonts}
\usepackage{bm,amssymb,amsmath,amsthm,wasysym,color,fullpage,setspace,multirow}
\usepackage{listings, xcolor}
\usepackage{pdfpages}
\newcommand{\vb}{\boldsymbol}
\newcommand{\vbh}[1]{\hat{\boldsymbol{#1}}}
\newcommand{\vbb}[1]{\bar{\boldsymbol{#1}}}
\newcommand{\vbt}[1]{\tilde{\boldsymbol{#1}}}
\newcommand{\vbs}[1]{{\boldsymbol{#1}}^*}
\newcommand{\vbd}[1]{\dot{{\boldsymbol{#1}}}}
\newcommand{\vbdd}[1]{\ddot{{\boldsymbol{#1}}}}
\newcommand{\by}{\times}
\newcommand{\tr}{{\rm tr}}
\newcommand{\cpe}[1]{\left[{#1} \times \right]}
\newcommand{\sfrac}[2]{\textstyle \frac{#1}{#2}}
\newcommand{\ba}{\begin{array}}
\newcommand{\ea}{\end{array}}
\renewcommand{\earth}{\oplus}
\newcommand{\sinc}{{\rm \hspace{0.5mm} sinc}}
\newcommand{\tf}{\tilde{f}}
\newcommand{\tbox}[1]{\noindent \fbox{\parbox{\textwidth}{#1}}}
\DeclareMathAlphabet{\mathpzc}{OT1}{pzc}{m}{it}
\definecolor{mylilas}{RGB}{170,55,241}
\definecolor{mygreen}{RGB}{0,168,45}

\title{ASE 389P.4 Methods of Orbit Determination \\ Homework 4}
\author{Corey L Marcus} \date{Thursday, March 23\textsuperscript{rd}}

%command to write C++ nicely
\def\CC{{C\nolinebreak[4]\hspace{-.05em}\raisebox{.4ex}{\tiny\bf ++}}}

%commands to include C++ code in appendix
\lstset { %
	language=C++,
	backgroundcolor=\color{black!5}, % set backgroundcolor
	basicstyle=\tiny,% basic font setting
}

\begin{document}
\onehalfspace
\maketitle

\abstract{This homework a coordinate frame shift from ECEF to ECI.}

\section{Introduction}

The IAU-76/FK5 transformation of position from ECEF to ECI was programmed in \CC. \\

\section{Problem 1}

The initial conditions (ECEF) in Table \ref{tb:prob1} were transformed to the ECI frame. The new coordinates and error are also shown.

\begin{table}[ht!]
	\centering
	\begin{tabular}{c|c|c|c}
		Item  & x & y & z \\ \hline
		ECEF Position [km]  & -28738.32184 & -30844.07232 & -6.718000 \\ \hline
		ECI Position [km]  & 19165.446254116094 & -37549.060839037964 & -41.043624939788835 \\ \hline
		Error [km]  & -0.0011063373531214893 & -0.00056470289564458653 & 1.4991506255057629e-05 \\ \hline

	\end{tabular}
	\caption{Results from Problem 1.}
	\label{tb:prob1}
\end{table}

There are four rotation matrices required to complete this transformation. These are reported in Figure \ref{fig:prob1}.

\begin{figure}[!htb]
	\centering
	\includegraphics[width=0.75\linewidth]{figs/pic.png}
	\label{fig:prob1}
	\caption{The four rotation matricies.}
\end{figure}

\newpage
\appendix
\section{Code}

\subsection{\texttt{main.cc}}
\lstinputlisting{../../src/HW4/main.cc}

\subsection{\texttt{Util.h}}
\lstinputlisting{../../src/HW4/Util.h}

\subsection{\texttt{Util.cc}}
\lstinputlisting{../../src/HW4/Util.cc}

\end{document}
