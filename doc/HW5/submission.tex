\documentclass[11pt]{article}
\usepackage{subfigure,wrapfig,graphicx,booktabs,fancyhdr,amsmath,amsfonts}
\usepackage{bm,amssymb,amsmath,amsthm,wasysym,color,fullpage,setspace,multirow}
\usepackage{listings, xcolor}
\usepackage{pdfpages}
\newcommand{\vb}{\boldsymbol}
\newcommand{\vbh}[1]{\hat{\boldsymbol{#1}}}
\newcommand{\vbb}[1]{\bar{\boldsymbol{#1}}}
\newcommand{\vbt}[1]{\tilde{\boldsymbol{#1}}}
\newcommand{\vbs}[1]{{\boldsymbol{#1}}^*}
\newcommand{\vbd}[1]{\dot{{\boldsymbol{#1}}}}
\newcommand{\vbdd}[1]{\ddot{{\boldsymbol{#1}}}}
\newcommand{\by}{\times}
\newcommand{\tr}{{\rm tr}}
\newcommand{\cpe}[1]{\left[{#1} \times \right]}
\newcommand{\sfrac}[2]{\textstyle \frac{#1}{#2}}
\newcommand{\ba}{\begin{array}}
\newcommand{\ea}{\end{array}}
\renewcommand{\earth}{\oplus}
\newcommand{\sinc}{{\rm \hspace{0.5mm} sinc}}
\newcommand{\tf}{\tilde{f}}
\newcommand{\tbox}[1]{\noindent \fbox{\parbox{\textwidth}{#1}}}
\DeclareMathAlphabet{\mathpzc}{OT1}{pzc}{m}{it}
\definecolor{mylilas}{RGB}{170,55,241}
\definecolor{mygreen}{RGB}{0,168,45}

\title{ASE 389P.4 Methods of Orbit Determination \\ Homework 5}
\author{Corey L Marcus} \date{Tuesday, April 6\textsuperscript{th}}

%command to write C++ nicely
\def\CC{{C\nolinebreak[4]\hspace{-.05em}\raisebox{.4ex}{\tiny\bf ++}}}

%commands to include C++ code in appendix
\lstset { %
	language=C++,
	backgroundcolor=\color{black!5}, % set backgroundcolor
	basicstyle=\tiny,% basic font setting
}

\begin{document}
\onehalfspace
\maketitle

\abstract{This homework demonstrates a high level orbit propagator}

\section{Introduction}

An orbit propagator was created in \CC. This propagator takes into account an EGM96 20x20 gravity model, solar radiation pressure, atmospheric drag, and 3\textsuperscript{rd} body effects of the sun and moon. \\

\section{Problem 1}

The dynamics jacobian, $A$, and measurement jacobian, $H$, were derived using matlab's symbolic toolbox. $A$ includes gravitational effects up to J2. Functions to evaluate these jacobians were written in \CC. They were evaluated at the initial time at the given true initial conditions. $A$ can be found in Eqn \eqref{eq:prob1A} and $H$ can be found in Eqn \eqref{eq:prob1H}. In all cases, the state used was position and velocity, coefficient of drag was not considered.

\begin{equation}
\label{eq:prob1A}
A = \begin{bmatrix}
	0 & 0 & 0 & 1 & 0 & 0 \\
	0 & 0 & 0 & 0 & 1 & 0 \\
	0 & 0 & 0 & 0 & 0 & 1 \\
	2.00020688165E-6 & 7.1151788333E-7 & 5.77760715E-9 & 0 & 0 & 0 \\
	7.1151788333E-7 & -9.1798909995 E-7 & 1.33364374E-9 & 0 & 0 & 0 \\
	5.77760715E-9 & 1.33364374E-9 & -1.08221778170E-6 & 0 & 0 & 0 \\
\end{bmatrix}
\end{equation}

\begin{equation}
\label{eq:prob1H}
H = \begin{bmatrix}
6.4778e-01  & 9.8367e-02 & -7.5545e-01     &       0       &     0     &       0 \\
-9.1059e-04 &  5.1371e-03 &  -1.9159e-04  & 6.4778e-01 &  9.8367e-02 &  -7.5545e-01 \\
\end{bmatrix}
\end{equation}

These matrices were differenced with their true values provided to produce relative difference matrices. These are shown in Eqn \eqref{eq:prob1Arel} and Eqn \eqref{eq:prob1Hrel}.

\begin{equation}
\label{eq:prob1Arel}
A = \begin{bmatrix}
0     &       0         &   0         &   0    &        0     &       0 \\
0          &  0        &    0     &       0     &       0       &     0 \\
0         &   0    &        0      &      0     &       0    &        0 \\
5.7993e-04&   1.5365e-03  & 4.2505e-01 &   1 &   1  &  1 \\
1.5368e-03  & 2.4958e-03  & 4.2648e-01 &   1  &  1  &  1 \\
4.2505e-01  & 4.2648e-01 &  1.0394e-03  &  1  &  1  &  1
\end{bmatrix}
\end{equation}

\begin{equation}
\label{eq:prob1Hrel}
H = \begin{bmatrix}
8.7030e-03 &  4.0309e-02  & 7.2897e-03    &        0   &         0    &        0 \\
1.3608e-02 &  3.9726e-03  & 9.2600e-01  & 8.7030e-03  & 4.0309e-02 &  7.2897e-03
\end{bmatrix}
\end{equation}

The base 10 logarithm of each element of the relative difference in A, Eqn \eqref{eq:prob1Arel}, was taken and plotted in a histogram. This is shown in Figure \ref{fig:prob1s}

\begin{figure}[!htb]
	\centering
	\includegraphics[width=0.75\linewidth]{figs/prob1.png}
	\label{fig:prob1}
	\caption{Histogram of the log10 of each of the elements of the relative difference in $A$.}
\end{figure}

\section{Problem 2}

\section{Problem 3}

\newpage
\appendix
\section{Code}

\subsection{\texttt{main.cc}}
\lstinputlisting{../../src/HW5/main.cc}

\subsection{\texttt{VehicleState.h}}
\lstinputlisting{../../src/HW5/Util.h}

\subsection{\texttt{VehicleState.cc}}
\lstinputlisting{../../src/HW5/Util.cc}

\subsection{\texttt{Util.h}}
\lstinputlisting{../../src/HW5/Util.h}

\subsection{\texttt{Util.cc}}
\lstinputlisting{../../src/HW5/Util.cc}

%plot matlab files now
\lstset { %
	language=matlab,
	backgroundcolor=\color{black!5}, % set backgroundcolor
	basicstyle=\tiny,% basic font setting
}

\subsection{\texttt{MeasurementSymbolic.m}}
\lstinputlisting{../../src/HW5/MeasurementSymbolic.m}

\subsection{\texttt{plotting.m}}
\lstinputlisting{../../src/HW5/plotting.m}

\end{document}
