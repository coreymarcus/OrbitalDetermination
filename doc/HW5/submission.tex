\documentclass[11pt]{article}
\usepackage{subfigure,wrapfig,graphicx,booktabs,fancyhdr,amsmath,amsfonts}
\usepackage{bm,amssymb,amsmath,amsthm,wasysym,color,fullpage,setspace,multirow}
\usepackage{listings, xcolor}
\usepackage{pdfpages}
\newcommand{\vb}{\boldsymbol}
\newcommand{\vbh}[1]{\hat{\boldsymbol{#1}}}
\newcommand{\vbb}[1]{\bar{\boldsymbol{#1}}}
\newcommand{\vbt}[1]{\tilde{\boldsymbol{#1}}}
\newcommand{\vbs}[1]{{\boldsymbol{#1}}^*}
\newcommand{\vbd}[1]{\dot{{\boldsymbol{#1}}}}
\newcommand{\vbdd}[1]{\ddot{{\boldsymbol{#1}}}}
\newcommand{\by}{\times}
\newcommand{\tr}{{\rm tr}}
\newcommand{\cpe}[1]{\left[{#1} \times \right]}
\newcommand{\sfrac}[2]{\textstyle \frac{#1}{#2}}
\newcommand{\ba}{\begin{array}}
\newcommand{\ea}{\end{array}}
\renewcommand{\earth}{\oplus}
\newcommand{\sinc}{{\rm \hspace{0.5mm} sinc}}
\newcommand{\tf}{\tilde{f}}
\newcommand{\tbox}[1]{\noindent \fbox{\parbox{\textwidth}{#1}}}
\DeclareMathAlphabet{\mathpzc}{OT1}{pzc}{m}{it}
\definecolor{mylilas}{RGB}{170,55,241}
\definecolor{mygreen}{RGB}{0,168,45}

\title{ASE 389P.4 Methods of Orbit Determination \\ Homework 5}
\author{Corey L Marcus} \date{Tuesday, April 6\textsuperscript{th}}

%command to write C++ nicely
\def\CC{{C\nolinebreak[4]\hspace{-.05em}\raisebox{.4ex}{\tiny\bf ++}}}

%commands to include C++ code in appendix
\lstset { %
	language=C++,
	backgroundcolor=\color{black!5}, % set backgroundcolor
	basicstyle=\tiny,% basic font setting
}

\begin{document}
\onehalfspace
\maketitle

\abstract{This homework demonstrates a high level orbit propagator}

\section{Introduction}

An orbit propagator was created in \CC. This propagator takes into account an EGM96 20x20 gravity model, solar radiation pressure, atmospheric drag, and 3\textsuperscript{rd} body effects of the sun and moon. \\

\section{Problem 1}

The dynamics jacobian, $A$, and measurement jacobian, $H$, were derived using matlab's symbolic toolbox. $A$ includes gravitational effects up to J2. Functions to evaluate these jacobians were written in \CC. They were evaluated at the initial time at the given true initial conditions. $A$ can be found in Eqn \eqref{eq:prob1A} and $H$ can be found in Eqn \eqref{eq:prob1H}. In all cases, the state used was position and velocity, coefficient of drag was not considered.

\begin{equation}
\label{eq:prob1A}
A = \left[\begin{array}{cccccc} 0 & 0 & 0 & 1.0 & 0 & 0\\ 0 & 0 & 0 & 0 & 1.0 & 0\\ 0 & 0 & 0 & 0 & 0 & 1.0\\ 2.0054e-6 & 7.1304e-7 & 5.8147e-9 & -6.02e-13 & 1.25e-13 & 5.0e-15\\ 7.1304e-7 & -9.1903e-7 & 1.3422e-9 & 1.25e-13 & -1.116e-12 & -2.1e-14\\ 5.8147e-9 & 1.3422e-9 & -1.0864e-6 & 5.0e-15 & -2.1e-14 & -5.74e-13 \end{array}\right]
\end{equation}

\begin{equation}
\label{eq:prob1H}
H = \begin{bmatrix}
6.4778e-01  & 9.8367e-02 & -7.5545e-01     &       0       &     0     &       0 \\
-9.1059e-04 &  5.1371e-03 &  -1.9159e-04  & 6.4778e-01 &  9.8367e-02 &  -7.5545e-01 \\
\end{bmatrix}
\end{equation}

These matrices were differenced with their true values provided to produce relative difference matrices. These are shown in Eqn \eqref{eq:prob1Arel} and Eqn \eqref{eq:prob1Hrel}.

\begin{equation}
\label{eq:prob1Arel}
A_{\text{rel. err.}} = \left[\begin{array}{cccccc} 0 & 0 & 0 & 0 & 0 & 0\\ 0 & 0 & 0 & 0 & 0 & 0\\ 0 & 0 & 0 & 0 & 0 & 0\\ 0.0031876 & 0.00060269 & 0.42136 & 0.7737 & 0.7731 & 0.7660\\ 0.00060241 & 0.0036308 & 0.42279 & 0.7731 & 0.7738 & 0.7669\\ 0.42136 & 0.42279 & 0.0028131 & 0.7660 & 0.7669 & 0.7737 \end{array}\right]
\end{equation}

\begin{equation}
\label{eq:prob1Hrel}
H_{\text{rel. err.}} = \begin{bmatrix}
8.7030e-03 &  4.0309e-02  & 7.2897e-03    &        0   &         0    &        0 \\
1.3608e-02 &  3.9726e-03  & 9.2600e-01  & 8.7030e-03  & 4.0309e-02 &  7.2897e-03
\end{bmatrix}
\end{equation}

The base 10 logarithm of each element of the relative difference in A, Eqn \eqref{eq:prob1Arel}, was taken and plotted in a histogram. This is shown in Figure \ref{fig:prob1A}.

\begin{figure}[!htb]
	\centering
	\includegraphics[width=0.75\linewidth]{figs/prob1.png}
	\caption{Histogram of the log10 of each of the elements of the relative difference in $A$.}
	\label{fig:prob1A}
\end{figure}

\section{Problem 2}

The position, velocity, and state transition matrix were computed up through 21600 seconds. The final position and velocity are reported below in Eqn \eqref{eq:prob2posvel}. The first three elements correspond to position in km and the final three correspond to velocity in km/sec.

\begin{equation}
	\label{eq:prob2posvel}
	\mathbf{X}(t_f) = \begin{bmatrix}
	-5153.7889765296113 \\
	-4954.4167389974418 \\
	-144.82758237664743 \\
	 5.1780598280691503 \\
	-5.3874917868480843 \\
	-0.21192904357397457	
	\end{bmatrix}
\end{equation}

The STM, $\Phi(21600,0)$, was also calculated and is reported in Eqn \eqref{eq:prob2Phi}. The STM was found based on effects from the central gravity, J2, J3, and drag. The base 10 logarithm of each element of the relative difference in $\Phi(21600,0)$, Eqn \eqref{eq:prob2Phirel}, was taken and plotted in a histogram. This is shown in Figure \ref{fig:prob2Phi}.

\begin{equation}
\label{eq:prob2Phi}
\Phi(21600,0) = \left[\begin{array}{cccccc} -48.009 & -12.678 & -0.15002 & 8891.7 & -48107.0 & -1712.1\\ 45.765 & 12.944 & 0.19746 & -7429.1 & 45160.0 & 1617.3\\ 1.8203 & 0.54049 & -0.8333 & -305.14 & 1822.6 & -450.59\\ -0.051079 & -0.012882 & -0.00013147 & 9.8854 & -50.567 & -1.8046\\ -0.046303 & -0.012461 & -0.00016794 & 7.9844 & -45.048 & -1.5698\\ -0.0013417 & -0.00038558 & 0.00055598 & 0.2215 & -1.2753 & -0.89069 \end{array}\right]
\end{equation}

\begin{equation}
\label{eq:prob2Phirel}
\Phi(21600,0)_{\text{rel. err.}} = \left[\begin{array}{cccccc} 0.2204 & 0.20928 & 0.15961 & 0.22551 & 0.21265 & 0.24763\\ 0.13344 & 0.12986 & 0.3275 & 0.14051 & 0.13456 & 0.10739\\ 0.053889 & 0.055462 & 0.091134 & 0.04709 & 0.051003 & 0.48326\\ 0.11072 & 0.13016 & 0.44657 & 0.084963 & 0.1116 & 0.084971\\ 0.27918 & 0.26819 & 0.070979 & 0.30903 & 0.28636 & 0.33279\\ 0.31646 & 0.31381 & 0.4046 & 0.35189 & 0.33957 & 0.063488 \end{array}\right]
\end{equation}

\begin{figure}[!htb]
	\centering
	\includegraphics[width=0.75\linewidth]{figs/prob2.png}
	\caption{Histogram of the log10 of each of the elements of the relative difference in $\Phi(21600,0)$.}
	\label{fig:prob2Phi}
\end{figure}

\section{Problem 3}

Using the measurement model, the predicted range and range rate measurements were generated. The pre-fit residuals are shown in Figure \ref{fig:prob3}. \\

\begin{figure}[!htb]
	\centering
	\includegraphics[width=0.75\linewidth]{figs/prob3.png}
	\caption{The pre-fit measurement residuals for range and range-rate.}
	\label{fig:prob3}
\end{figure}

The RMS values for these measurements were calculated and are shown in Equation \eqref{eq:prob3}. The first element corresponds to the range RMS values in km and the second element is range-rate RMS values in km/sec.

\begin{equation}
\label{eq:prob3}
\text{RMS} = \begin{bmatrix}
	1.4390e-02 \\
	3.6719e-04
\end{bmatrix}
\end{equation}


\newpage
\appendix
\section{Code}

\subsection{\texttt{main.cc}}
\lstinputlisting{../../src/HW5/main.cc}

\subsection{\texttt{VehicleState.h}}
\lstinputlisting{../../src/HW5/Util.h}

\subsection{\texttt{VehicleState.cc}}
\lstinputlisting{../../src/HW5/Util.cc}

\subsection{\texttt{Util.h}}
\lstinputlisting{../../src/HW5/Util.h}

\subsection{\texttt{Util.cc}}
\lstinputlisting{../../src/HW5/Util.cc}

%plot matlab files now
\lstset { %
	language=matlab,
	backgroundcolor=\color{black!5}, % set backgroundcolor
	basicstyle=\tiny,% basic font setting
}

\subsection{\texttt{MeasurementSymbolic.m}}
\lstinputlisting{../../src/HW5/MeasurementSymbolic.m}

\subsection{\texttt{plotting.m}}
\lstinputlisting{../../src/HW5/plotting.m}

\end{document}
