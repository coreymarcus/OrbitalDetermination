\documentclass[11pt]{article}
\usepackage{subfigure,wrapfig,graphicx,booktabs,fancyhdr,amsmath,amsfonts}
\usepackage{bm,amssymb,amsmath,amsthm,wasysym,color,fullpage,setspace,multirow}
\usepackage{listings, xcolor}
\usepackage{pdfpages}
\usepackage{siunitx}
\newcommand{\vb}{\boldsymbol}
\newcommand{\vbh}[1]{\hat{\boldsymbol{#1}}}
\newcommand{\vbb}[1]{\bar{\boldsymbol{#1}}}
\newcommand{\vbt}[1]{\tilde{\boldsymbol{#1}}}
\newcommand{\vbs}[1]{{\boldsymbol{#1}}^*}
\newcommand{\vbd}[1]{\dot{{\boldsymbol{#1}}}}
\newcommand{\vbdd}[1]{\ddot{{\boldsymbol{#1}}}}
\newcommand{\by}{\times}
\newcommand{\tr}{{\rm tr}}
\newcommand{\cpe}[1]{\left[{#1} \times \right]}
\newcommand{\sfrac}[2]{\textstyle \frac{#1}{#2}}
\newcommand{\ba}{\begin{array}}
\newcommand{\ea}{\end{array}}
\renewcommand{\earth}{\oplus}
\newcommand{\sinc}{{\rm \hspace{0.5mm} sinc}}
\newcommand{\tf}{\tilde{f}}
\newcommand{\tbox}[1]{\noindent \fbox{\parbox{\textwidth}{#1}}}
\DeclareMathAlphabet{\mathpzc}{OT1}{pzc}{m}{it}
\definecolor{mylilas}{RGB}{170,55,241}
\definecolor{mygreen}{RGB}{0,168,45}


\title{ASE 389P.4 Methods of Orbit Determination \\ Project Submission 1}
\author{Corey L Marcus} \date{Tuesday, April 6\textsuperscript{th}}

%command to write C++ nicely
\def\CC{{C\nolinebreak[4]\hspace{-.05em}\raisebox{.4ex}{\tiny\bf ++}}}

%commands to include C++ code in appendix
\lstset { %
	language=C++,
	backgroundcolor=\color{black!5}, % set backgroundcolor
	basicstyle=\tiny,% basic font setting
}

\begin{document}
\onehalfspace
\maketitle

\abstract{This homework demonstrates a high level orbit propagator}

\section{Introduction}

\section{Methodology}

\subsection{System Dynamics}

This subsection details the system dynamics used to model the spacecraft's orbit. All code was written in \CC for the languages greatly increased speed when compared to scripting languages such as MATLAB. Many of the algorithms used were sourced from the well-known text "Fundamentals of Astrodynamics and Applications" by David Vallado.

\subsubsection{Gravity Model}

An EGM-96 gravity model was used up to a 20x20 gravity field. The choice to use \CC was motivated primarily by the computational expense of this gravity model. A slower language such as MATLAB would have much longer propagation times for the same complexity gravity model.

\subsubsection{Drag Model}

A simple cannonball drag model was used due to the relatively small effect of drag on the spacecraft's orbit. The cross-sectional area of the spacecraft was approximated at \SI{22}{\meter\squared} and the coefficient of drag at 1.88.

\subsubsection{Solar Radiation Pressure Model}

Similar to atmospheric drag, a simple cannonball model was used to model Solar Radiation Pressure (SRP). SRP force is not modeled when the spacecraft is shielded from the sun by the Earth's shadow. The coefficient of drag for SRP was approximated as XXX.

\subsubsection{Luni-Solar Model}

Third body gravitational effects from the Sun and Moon were included. The acceleration due to each of these bodies was found according to a simple point-mass gravitational model. This simplification is justified due to the significant distance between the spacecraft and each of the third bodies. Algorithms to locate the Sun and Moon as a function of the current UTC time were found in \textit{Vallado}.

\subsection{Measurement Model}

Talk about time of light correction

\subsection{Estimation}

The estimation scheme centers around an Unscented Kalman Filter (UKF). The UKF is a nonlinear extension of the standard Kalman Filter. Its hallmark is approximation of covariances in the propagation and update phase through numerical propagation of a collection of sigma points chosen to conveniently represent the distribution of the prior. After propagation through time or into the measurement space, approximations of the mean and variance of the transformed distribution can be found by investigating the spacial distribution of the sigma points. \\

A ``$2n+1$" number of points were used along with a single tuning parameter $k=0.5$. This value was chosen so that all sigma points would be given the same weight. \\

Filter initialization is a challenge for many nonlinear extensions of the Kalman Filter. Improper initialization can quickly lead to filter divergence. A nonlinear least-squares optimization scheme was used to find the initial state estimate. This involved using Levenberg-Marquardt (LM) to select the initial state, $x_0$, such that the norm of the prefit residuals, Equation \eqref{eq:ICcost}, was minimized. To reduce the computational complexity, only the first 50 measurements were considered. In Equation \eqref{eq:ICcost} $f_a^b(\cdot)$ represents the nonlinear propagation of the satellite's state from the time of measurement $a$ to $b$, $h(\cdot)$ represents the measurement dynamics, and $z_i$ represents the $i$\textsuperscript{th} measurement. 

\begin{equation}
	\label{eq:ICcost}
	J(x_0) = \sum_{i=0}^{i=50} \left( z_i - h(f_{0}^{i}(x_0)) \right)^2 
\end{equation}

The resulting $x0$ is shown in Equation \eqref{eq:LM_IC}. The units are $km$ and $km/s$. The initial estimate covariance, $P_0$, was chosen arbitrarily to achieve a filter which converged for all test cases. This matrix is shown in Equation \eqref{eq:P0}. To correspond with the state, the units are $km^2$ and $km^2/s^2$. $diag \{a, b, c, ... \}$ is shorthand for a diagonal matrix with its arguments on the diagonal.

\begin{equation}
	\label{eq:LM_IC}
	x_0 = \begin{bmatrix}
	6980.3967323588 \\
	1619.61802198332 \\
	15.1399428739289 \\
	-1.66690187359566 \\
	7.2578409459164 \\
	0.261907498000759 \\	
	\end{bmatrix}
\end{equation}

\begin{equation}
	\label{eq:P0}
	P_0 = diag \{ 100, 100, 100, 0.01, 0.01, 0.01 \}
\end{equation}

\section{Deliverables}

\subsection{Prefit Residuals}

Figure \ref{fig:prefit} shows the prefit measurement residuals when all measurement types and sensors are considered.

\subsection{Postfit Residuals}

Figure \ref{fig:postfit} shows the postfit measurement residuals when all measurement types and sensors are considered.

\subsection{State Estimates at $\Delta V_1$}

To investigate the consitency of the estimates, case F (all data and all sensors) was designated as the best estimated trajectory. A transformation was found to map vectors into the Radial-Intrack-Crosstrack (RQW) coordinate frame described by the best estimate. For each case, the associated deviation from best-estimate and estimate covariance was transformed into the RQW frame. Figures \ref{fig:rad_cross}, \ref{fig:rad_in}, and \ref{fig:cross_in} showcase these deviations and covariances.
 
\section{Discussion}

From the Deliverables, we can form a variety of conclusions. We begin with a discussion of the prefit residuals. First we note that residuals at the beginning of the epoch are low, this indicates that our LM optimization scheme has most likely provided us an accurate initial state estimate. We also note that residuals are high immediately after a measurement gap; this is expected. Any estimation error immediately before a measurement gap is bound to grow during the nonlinear orbital propagation without any updates. Even if we had perfect estimation, error is bound to grow due to the unmodeled dynamics of our system. It is possible that our simplifications in dynamics modeling are contributing significantly to the growth of these residuals. We may be able to increase estimator performance by increasing the fidelity of our drag and SRP models. These large residuals also indicate that our estimate propagation to $\Delta V_1$ is likely to contain significant error.

Next, we discuss the postfit residuals.

\section{Conclusion}


\newpage
\appendix
\section{Code}

\subsection{\texttt{main.cc}}
\lstinputlisting{../../src/project/main.cc}

%%plot matlab files now
%\lstset { %
%	language=matlab,
%	backgroundcolor=\color{black!5}, % set backgroundcolor
%	basicstyle=\tiny,% basic font setting
%}


\end{document}
