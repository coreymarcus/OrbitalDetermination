\documentclass[11pt]{article}
\usepackage{subfigure,wrapfig,graphicx,booktabs,fancyhdr,amsmath,amsfonts}
\usepackage{bm,amssymb,amsmath,amsthm,wasysym,color,fullpage,setspace,multirow}
\usepackage{listings, xcolor}
\usepackage{pdfpages}
\newcommand{\vb}{\boldsymbol}
\newcommand{\vbh}[1]{\hat{\boldsymbol{#1}}}
\newcommand{\vbb}[1]{\bar{\boldsymbol{#1}}}
\newcommand{\vbt}[1]{\tilde{\boldsymbol{#1}}}
\newcommand{\vbs}[1]{{\boldsymbol{#1}}^*}
\newcommand{\vbd}[1]{\dot{{\boldsymbol{#1}}}}
\newcommand{\vbdd}[1]{\ddot{{\boldsymbol{#1}}}}
\newcommand{\by}{\times}
\newcommand{\tr}{{\rm tr}}
\newcommand{\cpe}[1]{\left[{#1} \times \right]}
\newcommand{\sfrac}[2]{\textstyle \frac{#1}{#2}}
\newcommand{\ba}{\begin{array}}
\newcommand{\ea}{\end{array}}
\renewcommand{\earth}{\oplus}
\newcommand{\sinc}{{\rm \hspace{0.5mm} sinc}}
\newcommand{\tf}{\tilde{f}}
\newcommand{\tbox}[1]{\noindent \fbox{\parbox{\textwidth}{#1}}}
\DeclareMathAlphabet{\mathpzc}{OT1}{pzc}{m}{it}
\definecolor{mylilas}{RGB}{170,55,241}
\definecolor{mygreen}{RGB}{0,168,45}

\title{ASE 389P.4 Methods of Orbit Determination \\ Homework 3}
\author{Corey L Marcus} \date{Thursday, March 11\textsuperscript{th}}

%command to write C++ nicely
\def\CC{{C\nolinebreak[4]\hspace{-.05em}\raisebox{.4ex}{\tiny\bf ++}}}

%commands to include C++ code in appendix
\lstset { %
	language=C++,
	backgroundcolor=\color{black!5}, % set backgroundcolor
	basicstyle=\tiny,% basic font setting
}

\begin{document}
\onehalfspace
\maketitle

\abstract{This homework demonstrates a calculation of the STM for basic orbits and basic estimation.}

\section{Introduction}

The propagator from previous homeworks was used to find the state transition matrix mapping deviations in a nominal trajectory at one time to another. \\

In addition, a basic estimation problem was solved.

\section{Problem 1}

\subsection{Problem 1-a}

We begin by propagating a orbit in a dimensionless system for 10 time units. The state at $i=1$ and $i=10$ is available in Equation \eqref{eq:stateproblem1a}. \\

The \texttt{runge\_kutta\_dopri5} (Runge Kutta Dormand-Prince 5) variable time-step numerical propagator from the well known \texttt{boost} {\CC} library was used. The absolute tolerance was set to 1E-16 and the relative to 3E-14.

\begin{equation}
	\label{eq:stateproblem1a}
	\begin{array}{lr}
		\underline{X}(t_1) = \begin{bmatrix}
			-0.83907 \\
			-0.54402 \\
			0.54402 \\
			-0.83907
		\end{bmatrix} & \underline{X}(t_{10}) = \begin{bmatrix}
			 0.86232 \\
			-0.50637 \\
			0.50637 \\
			0.86232
		\end{bmatrix}
	\end{array}
\end{equation}

\subsection{Problem 1-b}

Next, the initial conditions were perturbed slightly by the specified amount. The system was propagated with the same solver. The STM was also found through numerical integration. The perturbed state at $i=1$ and $i=10$ is available in Equation \eqref{eq:stateproblem1b}. The STM mapping from $t_0$ to $t_{1}$ and $t_{10}$ is shown in Equation \eqref{eq:STMproblem1b}.

\begin{equation}
	\label{eq:stateproblem1b}
	\begin{array}{lr}
		\underline{X}^*(t_1) = \begin{bmatrix}
			-0.83903 \\
			-0.54407 \\
			0.54408 \\
			-0.83904
		\end{bmatrix} & \underline{X}^*(t_{10}) = \begin{bmatrix}
			 0.86262 \\
			-0.50584 \\
			0.50585 \\
			0.86262
		\end{bmatrix}
	\end{array}
\end{equation}

\begin{equation}
	\label{eq:STMproblem1b}
	\begin{array}{c}
		\underline{\underline{\Phi}}(t_1, t_0) = \begin{bmatrix}
			-19.296 & -1.0006 & -1.5446 & -20.592 \\
			24.54  & 2.543 & 3.382 & 24.996 \\
			-26.628 & -1.247 & -2.086 & -27.541 \\
			-15.075 & -1.4571 & -2.0011 & -14.667 \\
		\end{bmatrix} \\
		\underline{\underline{\Phi}}(t_{10}, t_0) = \begin{bmatrix}
			-151.28 & -0.069643 & -0.57518 & -152.54 \\
			-260.23 &  0.88124  &  0.019132 &  -260.67 \\
			259.15 &  0.37464  &   1.2367  &  260.03 \\
			-152.13 &  0.36671 & -0.13883 & -151.64  
			
		\end{bmatrix}
	\end{array}
\end{equation}

\subsection{Problem 1-c}

To demonstrate that the STM is symplectic, we provide $\underline{\underline{\Phi}}(t_{10}, t_0)^{-1}$ in Equation \eqref{eq:STMinvproblem1c}. We multiply the STM with its inverse in Equation \eqref{eq:STMtimesInv} and show that their product is identity to the precision of the numerical integrator.

\begin{equation}
	\label{eq:STMinvproblem1c}
	\underline{\underline{\Phi}}(t_{10}, t_0)^{-1} = \begin{bmatrix}
		1.2367 & -0.13883 & 0.57518 & -0.019132 \\
		260.03 & -151.64 & 152.54 & 260.67 \\
		-259.15 & 152.13 & -151.28 & -260.23 \\
		-0.37464 & -0.36671 & -0.069643 & 0.88124
	\end{bmatrix}
\end{equation}

\begin{equation}
	\label{eq:STMtimesInv}
	\underline{\underline{\Phi}}(t_{10}, t_0) \underline{\underline{\Phi}}(t_{10}, t_0)^{-1} = \begin{bmatrix}
		1 & -2.1316e-14 & 1.2434e-14 & 2.8422e-14 \\
		-8.5265e-14 & 1 & 1.0658e-14 & -2.8422e-14 \\
		-5.6843e-14 & 7.1054e-14 & 1 & -8.5265e-14 \\
		-1.4211e-14 & 7.1054e-15 & 1.4211e-14 & 1
	\end{bmatrix}
\end{equation}

\subsection{Problem 1-d}

We find the perturbation from the nominal directly and compare to the peturbation estimated through the STM. This is done at $i=1$ and $i=10$. Results are shown in Tables \ref{tb:prob1d_1} and \ref{tb:prob1d_2}. \\

{\renewcommand{\arraystretch}{2}
\begin{table}[ht!]
	\centering
	\begin{tabular}{c|c}
		Quantity  & $i=1$ \\ \hline
		$\delta \underline{X}(t_i) = \underline{X}(t_i) -  \underline{X}^*(t_i)$ & $\left[-4.0431e-05 \quad 5.0376e-05 \quad -5.501e-05 \quad -3.0285e-05 \right]^T$ \\ \hline
		$\delta \underline{X}(t_i) = \underline{\underline{\Phi}}(t_i, t_0) \delta \underline{X}(t_0)$ & $\left[-4.0433e-05 \quad 5.0374e-05 \quad 7.5791e-06 \quad -5.5009e-05 \right]^T$ \\ \hline
		$\delta \underline{X}(t_i) - \underline{\underline{\Phi}}(t_i, t_0) \delta \underline{X}(t_0)$ & $\left[1.5866e-09 \quad 1.1069e-09 \quad -6.2589e-05 \quad 2.4724e-05 \right]^T$ 
	\end{tabular}
	\caption{Results from Problem 1-d at $i=1$.}
	\label{tb:prob1d_1}
\end{table}
}

{\renewcommand{\arraystretch}{2}
	\begin{table}[ht!]
		\centering
		\begin{tabular}{c|c}
			Quantity  & $i=10$ \\ \hline
			$\delta \underline{X}(t_i) = \underline{X}(t_i) -  \underline{X}^*(t_i)$ & $\left[-0.00030449 \quad -0.00052168 \quad 0.00051995 \quad -0.00030443 \right]^T$ \\ \hline
			$\delta \underline{X}(t_i) = \underline{\underline{\Phi}}(t_i, t_0) \delta \underline{X}(t_0)$ & $\left[-0.00030433 \quad -0.00052177 \quad 0.0010777 \quad 0.00052004	\right]^T$ \\ \hline
			$\delta \underline{X}(t_i) - \underline{\underline{\Phi}}(t_i, t_0) \delta \underline{X}(t_0)$ & $\left[-1.5836e-07 \quad 8.8785e-08 \quad -0.00055772 \quad -0.00082447 \right]^T$ 
		\end{tabular}
		\caption{Results from Problem 1-d at $i=10$.}
		\label{tb:prob1d_2}
	\end{table}
}

\newpage
\section{Problem 2}

A basic estimation problem was solved. The system is linear, static, and the mean and variance of the state's prior are known. A simple Kalman filter update was performed to find the state estimate, Equation \eqref{eq:estprob2}. The observation error can be estimated as $\underline{y} - \underline{\underline{H}} \hat{x}$ and is shown in Equation \eqref{eq:errestprob2}.

\begin{equation}
	\label{eq:estprob2}
	\hat{x} = 1.5000
\end{equation}

\begin{equation}
	\label{eq:errestprob2}
	\hat{\epsilon} = \begin{bmatrix}
		-0.5000 \\
		0.5000 \\
		-0.5000
	\end{bmatrix}
\end{equation}
\newpage
\appendix
\section{Code}

\subsection{\texttt{main.cc}}
\lstinputlisting{../../src/HW3/main.cc}

\subsection{\texttt{VehicleState.h}}
\lstinputlisting{../../src/HW3/VehicleState.h}

\subsection{\texttt{Util.h}}
\lstinputlisting{../../src/HW3/Util.h}

\subsection{\texttt{Estimators.h}}
\lstinputlisting{../../src/HW3/Estimators.h}

\subsection{\texttt{VehicleState.cc}}
\lstinputlisting{../../src/HW3/VehicleState.cc}

\subsection{\texttt{Util.cc}}
\lstinputlisting{../../src/HW3/Util.cc}

\subsection{\texttt{Estimators.cc}}
\lstinputlisting{../../src/HW3/Estimators.cc}

%%plot matlab files now
%\lstset { %
%	language=matlab,
%	backgroundcolor=\color{black!5}, % set backgroundcolor
%	basicstyle=\tiny,% basic font setting
%}
%
%\subsection{\texttt{plotting.m}}
%\lstinputlisting{../../src/HW1/plotting.m}


\end{document}
